\documentclass[12pt,]{article}
\usepackage{lmodern}
\usepackage{setspace}
\setstretch{1.2}
\usepackage{amssymb,amsmath}
\usepackage{ifxetex,ifluatex}
\usepackage{fixltx2e} % provides \textsubscript
\ifnum 0\ifxetex 1\fi\ifluatex 1\fi=0 % if pdftex
  \usepackage[T1]{fontenc}
  \usepackage[utf8]{inputenc}
\else % if luatex or xelatex
  \ifxetex
    \usepackage{mathspec}
  \else
    \usepackage{fontspec}
  \fi
  \defaultfontfeatures{Ligatures=TeX,Scale=MatchLowercase}
    \setmainfont[]{Times New Roman}
    \setsansfont[]{Times New Roman}
\fi
% use upquote if available, for straight quotes in verbatim environments
\IfFileExists{upquote.sty}{\usepackage{upquote}}{}
% use microtype if available
\IfFileExists{microtype.sty}{%
\usepackage{microtype}
\UseMicrotypeSet[protrusion]{basicmath} % disable protrusion for tt fonts
}{}
\usepackage[margin=1in]{geometry}
\usepackage{hyperref}
\PassOptionsToPackage{usenames,dvipsnames}{color} % color is loaded by hyperref
\hypersetup{unicode=true,
            colorlinks=true,
            linkcolor=Maroon,
            citecolor=Blue,
            urlcolor=Blue,
            breaklinks=true}
\urlstyle{same}  % don't use monospace font for urls
\usepackage{color}
\usepackage{fancyvrb}
\newcommand{\VerbBar}{|}
\newcommand{\VERB}{\Verb[commandchars=\\\{\}]}
\DefineVerbatimEnvironment{Highlighting}{Verbatim}{commandchars=\\\{\}}
% Add ',fontsize=\small' for more characters per line
\usepackage{framed}
\definecolor{shadecolor}{RGB}{248,248,248}
\newenvironment{Shaded}{\begin{snugshade}}{\end{snugshade}}
\newcommand{\AlertTok}[1]{\textcolor[rgb]{0.94,0.16,0.16}{#1}}
\newcommand{\AnnotationTok}[1]{\textcolor[rgb]{0.56,0.35,0.01}{\textbf{\textit{#1}}}}
\newcommand{\AttributeTok}[1]{\textcolor[rgb]{0.77,0.63,0.00}{#1}}
\newcommand{\BaseNTok}[1]{\textcolor[rgb]{0.00,0.00,0.81}{#1}}
\newcommand{\BuiltInTok}[1]{#1}
\newcommand{\CharTok}[1]{\textcolor[rgb]{0.31,0.60,0.02}{#1}}
\newcommand{\CommentTok}[1]{\textcolor[rgb]{0.56,0.35,0.01}{\textit{#1}}}
\newcommand{\CommentVarTok}[1]{\textcolor[rgb]{0.56,0.35,0.01}{\textbf{\textit{#1}}}}
\newcommand{\ConstantTok}[1]{\textcolor[rgb]{0.00,0.00,0.00}{#1}}
\newcommand{\ControlFlowTok}[1]{\textcolor[rgb]{0.13,0.29,0.53}{\textbf{#1}}}
\newcommand{\DataTypeTok}[1]{\textcolor[rgb]{0.13,0.29,0.53}{#1}}
\newcommand{\DecValTok}[1]{\textcolor[rgb]{0.00,0.00,0.81}{#1}}
\newcommand{\DocumentationTok}[1]{\textcolor[rgb]{0.56,0.35,0.01}{\textbf{\textit{#1}}}}
\newcommand{\ErrorTok}[1]{\textcolor[rgb]{0.64,0.00,0.00}{\textbf{#1}}}
\newcommand{\ExtensionTok}[1]{#1}
\newcommand{\FloatTok}[1]{\textcolor[rgb]{0.00,0.00,0.81}{#1}}
\newcommand{\FunctionTok}[1]{\textcolor[rgb]{0.00,0.00,0.00}{#1}}
\newcommand{\ImportTok}[1]{#1}
\newcommand{\InformationTok}[1]{\textcolor[rgb]{0.56,0.35,0.01}{\textbf{\textit{#1}}}}
\newcommand{\KeywordTok}[1]{\textcolor[rgb]{0.13,0.29,0.53}{\textbf{#1}}}
\newcommand{\NormalTok}[1]{#1}
\newcommand{\OperatorTok}[1]{\textcolor[rgb]{0.81,0.36,0.00}{\textbf{#1}}}
\newcommand{\OtherTok}[1]{\textcolor[rgb]{0.56,0.35,0.01}{#1}}
\newcommand{\PreprocessorTok}[1]{\textcolor[rgb]{0.56,0.35,0.01}{\textit{#1}}}
\newcommand{\RegionMarkerTok}[1]{#1}
\newcommand{\SpecialCharTok}[1]{\textcolor[rgb]{0.00,0.00,0.00}{#1}}
\newcommand{\SpecialStringTok}[1]{\textcolor[rgb]{0.31,0.60,0.02}{#1}}
\newcommand{\StringTok}[1]{\textcolor[rgb]{0.31,0.60,0.02}{#1}}
\newcommand{\VariableTok}[1]{\textcolor[rgb]{0.00,0.00,0.00}{#1}}
\newcommand{\VerbatimStringTok}[1]{\textcolor[rgb]{0.31,0.60,0.02}{#1}}
\newcommand{\WarningTok}[1]{\textcolor[rgb]{0.56,0.35,0.01}{\textbf{\textit{#1}}}}
\usepackage{longtable,booktabs}
\usepackage{graphicx,grffile}
\makeatletter
\def\maxwidth{\ifdim\Gin@nat@width>\linewidth\linewidth\else\Gin@nat@width\fi}
\def\maxheight{\ifdim\Gin@nat@height>\textheight\textheight\else\Gin@nat@height\fi}
\makeatother
% Scale images if necessary, so that they will not overflow the page
% margins by default, and it is still possible to overwrite the defaults
% using explicit options in \includegraphics[width, height, ...]{}
\setkeys{Gin}{width=\maxwidth,height=\maxheight,keepaspectratio}
\IfFileExists{parskip.sty}{%
\usepackage{parskip}
}{% else
\setlength{\parindent}{0pt}
\setlength{\parskip}{6pt plus 2pt minus 1pt}
}
\setlength{\emergencystretch}{3em}  % prevent overfull lines
\providecommand{\tightlist}{%
  \setlength{\itemsep}{0pt}\setlength{\parskip}{0pt}}
\setcounter{secnumdepth}{5}
% Redefines (sub)paragraphs to behave more like sections
\ifx\paragraph\undefined\else
\let\oldparagraph\paragraph
\renewcommand{\paragraph}[1]{\oldparagraph{#1}\mbox{}}
\fi
\ifx\subparagraph\undefined\else
\let\oldsubparagraph\subparagraph
\renewcommand{\subparagraph}[1]{\oldsubparagraph{#1}\mbox{}}
\fi

%%% Use protect on footnotes to avoid problems with footnotes in titles
\let\rmarkdownfootnote\footnote%
\def\footnote{\protect\rmarkdownfootnote}

%%% Change title format to be more compact
\usepackage{titling}

% Create subtitle command for use in maketitle
\newcommand{\subtitle}[1]{
  \posttitle{
    \begin{center}\large#1\end{center}
    }
}

\setlength{\droptitle}{-2em}

  \title{\vspace{1cm}Ex-post Survey Data Harmonization Toolbox*\footnote{*Corresponding address: \href{mailto:kolczynska.1@osu.edu}{\nolinkurl{kolczynska.1@osu.edu}}. Acknowledgments: I am grateful to all those generous people who invest their time into open-source software and reproducible research.}\vspace{0.5cm}\\}
    \pretitle{\vspace{\droptitle}\centering\huge}
  \posttitle{\par}
    \author{Marta Kołczyńska\\
Institute of Philosophy and Sociology, Polish Academy of Sciences}
    \preauthor{\centering\large\emph}
  \postauthor{\par}
      \predate{\centering\large\emph}
  \postdate{\par}
    \date{29 March, 2019\\
~\\}

\usepackage[english]{babel}
\usepackage[T1]{fontenc}
\usepackage{float}
\usepackage{lmodern}
\usepackage{graphicx}
\usepackage{url}
\usepackage{array}
\usepackage{setspace}
\usepackage{graphics}
%\usepackage{amssymb}
\usepackage{amsmath}
\usepackage{longtable}
\usepackage{natbib}
\usepackage{pdflscape}
\usepackage{caption}
\usepackage{subcaption}
\usepackage{fullpage}
\usepackage{multicol}
\usepackage{dcolumn}
\usepackage{lscape}
\usepackage{array}
\usepackage{color}
\usepackage{setspace}

\usepackage{cleveref}[2012/02/15]
\usepackage{fullpage}
\usepackage[charter]{mathdesign}
\usepackage{bm}
\usepackage{tabu}
\usepackage{wrapfig}

\usepackage[T1]{fontenc}
%\usepackage[polish]{babel}
\usepackage[utf8]{inputenc}

\crefformat{footnote}{#2\footnotemark[#1]#3}
\usepackage{color}
\usepackage{hyperref}

\newenvironment{tightcenter}{%
  \setlength\topsep{0pt}
  \setlength\parskip{0pt}
  \begin{center}
}{%
  \end{center}
}

%\setstretch{1.1}
% Dutch style of paragraph formatting
\parskip 5pt
\makeatletter
% Separation between lines
\doublerulesep 1pt
\doublespacing

\usepackage[table]{xcolor}

\definecolor{lightred}{rgb}{0.7,0,0}
\definecolor{darkgreen}{rgb}{0,0.8,0}
\definecolor{lightblue}{rgb}{0,0,0.7}

\hypersetup{colorlinks,
  linkcolor = lightred,
  filecolor = lightred,
  urlcolor = lightblue,
  citecolor = lightred}

% Spaces in bibliography
\let\oldthebibliography=\thebibliography
\let\endoldthebibliography=\endthebibliography
\renewenvironment{thebibliography}[1]{%
  \begin{oldthebibliography}{#1}%
    \setlength{\parskip}{0ex}%
    \setlength{\itemsep}{0.1ex}%
  }%
  {%
  \end{oldthebibliography}%
}

\usepackage{array}
\newcolumntype{L}[1]{>{\raggedright\let\newline\\\arraybackslash\hspace{0pt}}m{#1}}
\newcolumntype{C}[1]{>{\centering\let\newline\\\arraybackslash\hspace{0pt}}m{#1}}
\newcolumntype{R}[1]{>{\raggedleft\let\newline\\\arraybackslash\hspace{0pt}}m{#1}}

\setlength{\parindent}{0pt} % no indentation
\usepackage{booktabs}
\usepackage{longtable}
\usepackage{array}
\usepackage{multirow}
\usepackage{wrapfig}
\usepackage{float}
\usepackage{colortbl}
\usepackage{pdflscape}
\usepackage{tabu}
\usepackage{threeparttable}
\usepackage{threeparttablex}
\usepackage[normalem]{ulem}
\usepackage{makecell}
\usepackage{xcolor}

\usepackage{dcolumn}
\usepackage{color}

\begin{document}
\maketitle
\begin{abstract}
\noindent\setstretch{1}Ex-post harmonization of survey data creates new opportunities for research by extending the geographical and/or time coverage of analyses. While an increasing number of scholars combine different survey projects to analyze them as a single dataset, there is substantial variation in the understanding of what ex-post harmonization entails, and with regard to the methods of data processing and documentation. In this paper I propose a procedure and a set of simple tools for the exploration, recoding, and documentation of harmonization of survey data, relying on the programming language R and Excel spreadsheets - both common software among social scientists. This approach combines automation for improved reproducibility and efficiency, with human decision-making that allows for flexibility necessary in dealing with the variation and diverse standard observed in survey datasets. The resulting documentation of the harmonization process is user-friendly and readible without the knolwedge of any programming language. Variables on trust in institutions from three cross-national survey projects serve as an illustrate of the proposed workflow.\vspace{.8cm}
\end{abstract}

\clearpage

\renewcommand{\baselinestretch}{0.5}\normalsize
\tableofcontents
\renewcommand{\baselinestretch}{1.5}\normalsize

\clearpage

\hypertarget{abstract}{%
\section{Abstract}\label{abstract}}

Ex-post harmonization of survey data creates new opportunities for research by extending the geographical and/or time coverage of analyses. While an increasing number of scholars combine different survey projects to analyze them as a single dataset, there is substantial variation in the understanding of what ex-post harmonization entails, and with regard to the methods of data processing and documentation. In this paper I propose a procedure and a set of simple tools for the exploration, recoding, and documentation of harmonization of survey data, relying on the programming language R and Excel spreadsheets - both common software among social scientists. This approach combines automation for improved reproducibility and efficiency, with human decision-making that allows for flexibility necessary in dealing with the variation and diverse standard observed in survey datasets. The resulting documentation of the harmonization process is user-friendly and readible without the knolwedge of any programming language. Variables on trust in institutions from three cross-national survey projects serve as an illustrate of the proposed workflow.

\hypertarget{introduction}{%
\section{Introduction}\label{introduction}}

With extensive data archives in place and further archiving efforts in progress, the focus on combining and merging of existing data in order to build on what had already been done seems to be the optimal strategy and logical next step of developing the research infrastructure, and has already received supportive recognition (Burkhauser \& Lillard, \protect\hyperlink{ref-Burkhauser2005}{2005}).

International, and even multi-wave, survey projects are a major advancement in the social science infrastructure for cross-national research, yet their country and time coverage remains necessarily limited by funding availability, organizational conditions, and PIs' interests. Ex-post harmonization of survey data promises to overcome these limitations to create larger datasets with more global coverage. At the same time, ex-post harmonization can create country time series necessary for longitudinal analyses, which enables stronger tests of theoretical mechanisms than cross-country comparisons. Thus, harmonization of existing data in order to maximally exploit their potential for research is one of the major challenges in the social sciences. While the promises of new research opportunities are alluring to many, the associated challenges are multi-faceted - including technical, logistical, methodological, and substantive - and not yet well understood. This paper addresses one of these challenges, concerning the organization and documentation of the ex-post survey data harmonization process in a way that enables reproducibility of all data processing, as well as the cumulative character of harmonization efforts.

The number of projects relying on combined data from different cross-national survey projects is increasing. These projects range from small initiatives where the multi-projects dataset is created for the purposes of a single publication or dissertation (e.g., Christmann, \protect\hyperlink{ref-Christmann2018}{2018}; Christmann \& Torcal, \protect\hyperlink{ref-Christmann2018a}{2018}; Mauk, \protect\hyperlink{ref-Mauk2019}{2019}) to large projects performing ex-post survey data harmonization not - or not only - for own substantive research, but in order to improve the social science research infrastructure (Bekkers et al., \protect\hyperlink{ref-Bekkers2015}{2015}; Klassen, \protect\hyperlink{ref-Klassen2018}{2018}; Slomczynski \& Tomescu-Dubrow, \protect\hyperlink{ref-Slomczynski2018}{2018}). Correspondingly, the effort invested in the creation and sharing of the documentation of the harmonization process is characterized by substantial variation.

The diversity of approaches to documenting the harmonization process reflects the lack of established and followed standards for computational reproducibility, i.e.~the ability to re-create the results of published research using materials - data and code - provided by the authors of the original study (cf.~Liu \& Salganik, \protect\hyperlink{ref-Liu2019}{2019}). The value of published research largely depends on its reproducilibity (Buckheit \& Donoho, \protect\hyperlink{ref-Buckheit1995}{1995}). In addition to data and code, researchers are starting to recognize the role of the programming environment in reproducibility of results both currently, and in the near and more distant future. For example, Liu \& Salganik (\protect\hyperlink{ref-Liu2019}{2019}) propose to use software containers (such as Docker) to standardize the software and cloud computing (such as Amazon Web Services) to standardize the hardware, especially in computation-intensive analyses.

Reproducibility not only concerns research publications, but also processed data, such as those resulting from ex-post survey data harmonization products, although in this case it might be more accurately called `data processing reproduciblity', thus excluding data modeling.

In this paper I propose a procedure and a set of simple tools for the exploration, recoding, and documentation of harmonization of survey data, relying on crosswalks for mapping one coding scheme onto another. The proposed approach includes automated steps that ensure reproducibility and efficiency of data processing, with human decision-making to integrate methodological expertise and domain knowledge, and enable flexibility necessary in dealing with the variation and diverse standard observed in survey datasets.
The product of the harmonization process is its documentation in form of crosswalk tables that map (1) source variables to target variables and (2) source values to target values. The human-friendly character and readibility of crosswalks enables the verification and reproduction of the harmonization process.

The example presented in this paper uses the programming language R and Excel spreadsheets - both common software among social scientists. The general framework combining recoding syntax and spreadsheets is software-agnostic and can be used with any programming language. Survey variables on trust in institutions from three cross-national survey projects - the European Social Survey, the European Values Study, the European Quality of Life Survey - serve as an illustration of the proposed workflow.

This work builds on the experiences of four inter-related projects: (1) the \emph{Democratic Values and Protest Behavior: Data Harmonization, Measurement Comparability, and Multi-Level Modeling} project funded by the (Polish) National Science Centre (2012/06/M/HS6/00322, 2013-2016), its continuation \emph{Survey Data Recycling: New Analytic Framework, Integrated Database, and Tools for Cross-national Social, Behavioral and Economic Research} funded by the U.S. National Science Foundation (2017-2021, PTE Federal award 1738502), (3) \emph{New Approach to Analyses of the Relationship between Democracy and Trust: Comparing European Countries Using Quantitative and Qualitative Methodology} funded by the (Polish) National Science Centre (2012/05/N/HS6/03886, 2013-2015), (4) \emph{Stratified modernity, trust in state institutions and democratic change in cross-national perspective: using harmonized survey data from 116 countries} project supported by the Silverman Research Support Award from the Department of Sociology, The Ohio State University (2017), and (5) \emph{Effects of status inconsistency on political values, attitudes and behavior: a cross-national analysis with survey data harmonized ex post}, funded by the Institute of Philosophy and Sociology, Polish Academy of Science (2017/S/05, 2018).

The proposed procedures are substantially simplified compared to those implemented in the aforementioned projects in the spirit of accessibility and reproducibility, and the harmonization tools were developed independently. Altogether, the workflow proposed in this paper is sufficiently simple that it can be successfully implemented by a single person or a small team of programming non-specialists, and powerful enough that it can handle large amounts of data and harmonization situations of moderate complexity, with highly complex cases possible to accomodate after some modifications.

\hypertarget{what-is-ex-post-survey-data-harmonization}{%
\section{What is ex-post survey data harmonization?}\label{what-is-ex-post-survey-data-harmonization}}

Ex-post (or retrospective) data harmonization refers to procedures applied to already collected data to improve the comparability and inferential equivalence of measures collected by different studies (Fortier et al., \protect\hyperlink{ref-Fortier2017a}{2017}). In the case of ex-post survey data harmonization, the procedures are applied to survey data sets that were not intended for joint analysis, in order to integrate them into a single dataset that can be meaningfully employed in substantive analyses. The harmonization process is simultaneously theory-informed and data-driven. Theories provide the concepts to be measured, but data availability to a large extent determines what ends up being measured and how. The following steps can be distinguished (cf.~Granda \& Blasczyk, \protect\hyperlink{ref-Granda2016}{2016}; Fortier et al., \protect\hyperlink{ref-Fortier2017a}{2017}; Kołczyńska, \protect\hyperlink{ref-Kolczynska2019}{2019}; Wolf, Schneider, Behr, \& Joye, \protect\hyperlink{ref-Wolf2016}{2016}):

\begin{enumerate}
\def\labelenumi{(\arabic{enumi})}
\tightlist
\item
  concept definition:

  \begin{enumerate}
  \def\labelenumii{\alph{enumii}.}
  \tightlist
  \item
    defining the concept of interest guided by the theoretical framework and hypotheses;
  \item
    based on this definition, developing a preliminary coding scheme or choosing a coding scale for the harmonized (target) variable;
  \end{enumerate}
\item
  data preparation:

  \begin{enumerate}
  \def\labelenumii{\alph{enumii}.}
  \tightlist
  \item
    gathering available surveys (data and documentation) that meet the requirements regarding the topics, target population and representativeness, and potentially others;
  \item
    exploring the methodological variation they represent with regard to the design of the survey items of interest and the overall survey process;
  \item
    describing surveys in terms of methodology, including quality (and potentially exclude some surveys on this basis), and constructing survey quality indicators;
  \item
    identifying relevant items in the gathered surveys that correspond to the target concept(s);
  \item
    describing the methodological variation in the design of the selected survey items given the research of survey methodology and effects of item design on respondents' answers (potentially exclude some items in this step);
  \item
    identifying relevant dimensions of variation between the survey items to be captured by harmonization control variables;
  \item
    adjusting the coding scheme or scale of the harmonized (target variable) based on the observed variation in the survey items;
  \end{enumerate}
\item
  harmonization:

  \begin{enumerate}
  \def\labelenumii{\alph{enumii}.}
  \tightlist
  \item
    transforming source variables into the target variable(s) using the coding scheme;
  \item
    constructing harmonization control variables to capture the properties of source variables that would be lost in the process of recoding;
  \end{enumerate}
\item
  checking the target variable for errors and documentation of the whole process.
\end{enumerate}

\hypertarget{ex-post-survey-data-harmonization-projects-a-non-exhaustive-review}{%
\subsection{Ex-post survey data harmonization projects: A non-exhaustive review}\label{ex-post-survey-data-harmonization-projects-a-non-exhaustive-review}}

Researchers increasingly combine data from multiple survey projects to fill in the gaps in geographical coverage, or to increase the presence of typically underrepresented countries, e.g.~autocracies or countries from the Global South. Identifying projects that turn to ex-post survey data harmonization is not easy, because the term ``ex-post harmonization'' is not often used. This section briefly describe the approaches taken in a non-representative sample of ex-post survey data harmonization projects. The number of publications relying on survey data harmonized ex-post is difficult to know, because the term ``ex-post survey data harmonization'' is not widely used as the name of this procedure. An alternative term, ``integrative data analysis'', proposed by Curran \& Hussong (\protect\hyperlink{ref-Curran2009}{2009}), has not caught on either.

Based on a non-systematic review of ex-post survey data harmonization projects, two broad categories can be distinguished: small-scale projects where harmonization is performed for the purposes of a single paper or dissertation, and large-scale projects that enable multiple different analyses with the general aim of improving the social science research infrastructure.

\hypertarget{small-scale-projects}{%
\subsubsection{Small-scale projects}\label{small-scale-projects}}

The two small-scale projects I identified, ex-post survey data harmonization provided data for dissertation research. The first project (Christmann, \protect\hyperlink{ref-Christmann2018}{2018}; Christmann \& Torcal, \protect\hyperlink{ref-Christmann2018a}{2018}) harmonized data from several cross-national and national survey projects. The selection criterion for the projects was a particular design of the satisfaction with democracy question, which was the key variable in the analysis: only surveys where a four-point response scale was used were harmonized (Christmann, \protect\hyperlink{ref-Christmann2018}{2018}; Christmann \& Torcal, \protect\hyperlink{ref-Christmann2018a}{2018}). Replication materials that accompany the paper, available from Harvard's Dataverse, include the final harmonized data file and code for conducting the analysis, but no harmonization syntax (\url{https://doi.org/10.7910/DVN/EU541C}).

The second project deals with political trust and democratic values, and relies on data from six cross-national survey projects from a hundred countries and almost global coverage (Mauk, \protect\hyperlink{ref-Mauk2019}{2019}). Replication materials, also stored on Dataverse, include a recoding syntax file with all the code necessary to create the harmonized variables used in subsequent analyses. The harmonized dataset is not provided (\url{https://doi.org/10.7910/DVN/IIVAJM}).

\hypertarget{infrastructure-projects}{%
\subsubsection{Infrastructure projects}\label{infrastructure-projects}}

There are at least three large-scale projects that perform ex-post survey data harmonization. The first one is the Democratic Values and Protest Behavior (DVPB) project (Slomczynski \& Tomescu-Dubrow, \protect\hyperlink{ref-Slomczynski2018}{2018}, p. @Slomczynskietal2016), which harmonized selected variables on political participation, political trust, and basic socio-demographics, from 22 major international survey projects between 1966 and 2013. The data are publicly available via Harvard's Dataverse\footnote{The data and documentation from the Democratic Values and Protest Behavior (DVPB) project are available via Dataverse at \url{https://doi.org/10.7910/DVN/VWGF5Q}}
The project, now completed, used a suite of open-source tools for extracting source data, applying the harmonization procedures, and outputting data files in a format usable by social scientists. The programming environment is based on scripting tools, while data processing occurs in a MySQL database (Powałko \& Kołczyńska, \protect\hyperlink{ref-Powako2016}{2016}). The publicly available documentation consists of codebooks, recode syntax in SQL and lists of source variable names selected for harmonization from each dataset.
The continuation of this initiative, the Survey Data Recycling project (dataharmonization.org), will extend the set of harmonized variable to also include social capital and well-being indicators, and update the data up until 2017.

The second project, Human Understanding Measured Across National Surveys (HUMANS, humansurveys.org) uses data from 19 cross-national and national survey projects, many of them the same as in the previous project, and harmonized variables on social trust, satisfaction with democracy, support for democracy, and perceived electoral integrity, as well as basic socio-demographics. The data are also freely available.\footnote{The data and documentation from the Human Understanding Measured Across National Surveys (HUMANS, humansurveys.org) projects are available at \url{https://dataverse.harvard.edu/dataverse/humansurveys}.}
In terms of documentation, the codebook (Klassen, \protect\hyperlink{ref-Klassen2018}{2018}) provides the recodes from source to target coding schemes, names of source variables, and names of source data files, all in a PDF document.

The third initiative, the Harmonized Trust Database created by the Global Trust Research Consortium (GTRC) (globaltrustresearch.wordpress.com), contains data from 79 national and cross-national survey projects, covering 155 countries since 1953. The project Open Science Framework profile\footnote{The Harmonized Trust Database created by the Global Trust Research Consortium (GTRC) (globaltrustresearch.wordpress.com) makes the documentation available via OSF at \url{https://osf.io/qfv76/}} provides a PDF codebook (Sandberg \& Bekkers, \protect\hyperlink{ref-Sandberg2018}{2018}) with names of source projects, but without data file versions or source variable names. The harmonized data are not publicly available.

The three projects have very similar goals and scopes - in terms of the type of source data and the substantive interest in harmonizing particular variables - and yet, their approaches to documenting the harmonization, and the extent the harmonization is reproducible, are very different. While the DVPB project published voluminous documentation, in practice the exact replication of all harmonization procedures would most likely not be straightforward, despite the natural language-resembling nature of SQL, due to complications with re-creating the software environment. The HUMANS project published source and target variables, but the recoding schema is provided in a PDF document - not the most machine-readable format. GTRC provides neither the recode schemas, nor the resulting data.

The three large project described in this section also have one other thing in common: with the data and documentation made public by these projects it is not possible to extend the dataset by adding extra harmonized target variables. The reason is that the two projects that provide the harmonized data - DVPB and HUMANS - do not include original respondent (case) IDs from the source data files, so even though it is possible to construct additional variables separately, it is impossible to match them to the already harmonized data.

\hypertarget{the-proposed-workflow}{%
\section{The proposed workflow}\label{the-proposed-workflow}}

In the proposed workflow harmonization documentation is a product of the harmonization process itself. This section describes the harmonization steps and associated tools, to provide an overview of the whole process.

The procedure starts with all the source data files downloaded to a single location, and their origin and versions noted. Next, each source data file is imported to R, and inspected to create the following technical variables: source case ID, target case ID (row number in the data file), survey project, survey wave/round identifier, survey year, survey country, and case weights.

In terms of data processing, the harmonization work involves working with data at different levels. For each of these steps, a corresponding table is created on the basis of the source data, and exported to Excel for manual mapping. The resulting crosswalks are used in the next step of harmonization, and at the same time serve as documentation.

Crosswalks are commonly used tools for mapping of one schema onto the other. They are most useful when the source schema can be unambiguously translated into the target schema. This is rarely the case as exemplified by the many examples of class schemas developed in the social sciences. Also in the process of ex-post survey data harmonization, in some cases the mapping is not straightforward and decisions, sometimes arbitrary, need to be made. Like in other cases, also here, ``{[}T{]}he key to a successful metadata crosswalk is intelligent flexibility'' (Hillmann \& Westbrooks, \protect\hyperlink{ref-Hillmann2004}{2004}, p. 91). This is why the mapping is performed manually. Given the typical number of variables harmonized, this should not be an excessive effort.

Step 1: Selection of source variables for harmonization,\\
Step 2: Mapping source values to target values,\\
Step 3: Recording characteristics of source items.

The first step is at the variable level, and involves identifying variables for harmonization in the source data files, and assigning a standardized target variable name to each source variable of interest. To do this, a table with a list of all source variables is created on the basis of metadata from the source data files, and includes variable names, labels, as well as values and value labels and their corresponding frequencies. Each variable corresponds to a single row in the resulting table, which I refer to as the \emph{codebook}. The codebook is then exported to Excel (or other spreadsheet program), where variable lables are filtered to inentify candidates for source variables. In some cases, for example is variable labels are too short or otherwise uninformative - the original survey documentation needs to be consulted. After all source variables corresponding to the concept of interest - in this case trust in institutions - are identified via filter searches or otherwise, the table is imported back into R.

The second step is at the level of individual values of the source variables, and requires mapping these source values onto a common coding scheme of the respective target variable. Subsets of source data files are selected that include only the variables tagged in Step 1. For those variables, a \emph{cross-walk table} is created, i.e., a table similar to the codebook, but where each source value corresponds to one row. This table is then exported to a spreadsheet, where each source value is assigned a target value on the basis of the common coding scheme. This step also includes identifying missing value codes (such as negative numbers or multiples of 8 or 9) as missing. Next, the cross-walk table is imported back to R, where the source and target values for each source variable are mapped resulting in the harmonized variables.

The third step is at the data file level, and refers to the coding of properties of the source variables that are worth preserving because of methodological reasons, such as the length or direction of the original response scales. Typically, these properties vary between, but not within, survey projects. For example, in the European Social Survey, questions about trust in institutions are accompanied by 11-point scales ranging from 0 (No trust at all) to 10 (Completely trust) (ESS (European Social Survey), \protect\hyperlink{ref-ESSEuropeanSocialSurvey2016b}{2016}).

\hypertarget{illustration-trust-in-institutions}{%
\section{Illustration: Trust in institutions}\label{illustration-trust-in-institutions}}

The illustrative example deal with trust in institutions items and their basic correlates in three main cross-national survey projects in Europe: the European Social Survey, the European Values Study, and the European Quality of Life Survey. The data are available from the project website, the GESIS archive, and the UK Data Archive, respectively. Table \ref{tab:project-table} presents basic information about these survey projects.

Trust in political and other institutions is a common item in many cross-national surveys, and appears in all waves of all three projects. Each project includes questions about trust in a different set of institutions. Further, while in all three projects questionnaires tend to be relatively stable from wave to wave, the sets of questions have seen some changes, in particular new items added, and not always a given item was asked in all countries in a given wave. Finally, in each project the design of the trust items is slightly different. Major differences include the length and direction of the response scale, while other differences - response scale polarity or details of the question wording - seem to be of lesser importance for the distribution of the answers (Kołczyńska \& Slomczynski, \protect\hyperlink{ref-KolczynskaSlomczynski2018}{2018}).

For researchers interested in analyzing trust in a comparative perspective, it is important to know which projects include which trust items in which waves and countries, i.e., how rich a dataset they can count on. Like in all situations where the availability of variables for analysis is not clear from the start, this is an exploratory situation.

\begin{table}[t]

\caption{\label{tab:project-table}Description of the survey projects.}
\fontsize{11}{13}\selectfont
\begin{tabu} to \linewidth {>{\raggedright\arraybackslash}p{20em}>{\centering\arraybackslash}p{4em}>{\centering\arraybackslash}p{4em}>{\centering\arraybackslash}p{4em}>{\centering}X}
\toprule
\multicolumn{1}{c}{ } & \multicolumn{3}{c}{Number of} & \multicolumn{1}{c}{ } \\
\cmidrule(l{3pt}r{3pt}){2-4}
Project name & waves & data files & surveys & Years\\
\midrule
\rowcolor{gray!6}  European Social Survey (ESS) & 8 & 1 & 195 & 2002-2017\\
European Quality of Life Survey (EQLS) & 4 & 1 & 126 & 2003-2016\\
\rowcolor{gray!6}  European Values Study (EVS) & 4 & 2 & 140 & 1981-2017\\
\bottomrule
\end{tabu}
\end{table}

\hypertarget{step-0-preparation-and-coding-of-technical-variables}{%
\subsection{Step 0: Preparation and coding of technical variables}\label{step-0-preparation-and-coding-of-technical-variables}}

In the proposed schema, survey projects correspond to the organizational structures that public data under the same brand, such as the World Values Survey or the European Social Survey. Some of the projects have multiple waves of data collection. Each wave consists of national surveys, identified as a survey carried out in a given country and project wave. National surveys are identified as \emph{project*wave*country}, which typically is equivalent to \emph{project*year*country}. Most often the national survey level is equivalent to the sample level, i.e., a national survey is administered to respondents who are part of the same sample, with the sample representative for the entire population of a given country.\footnote{In some cases, in more than one sample is drawn in one country for historical, administrative, or cultural reasons, e.g., in Germany (East and West Germany) or in Belgium (Wallonia and Flanders). These sub-national samples can be either treated as separate surveys, or combined into a single survey with appropriate weights. In the projects included in this analysis the problem of sub-national samples does not exist, but it is signaled for the sake of clarity.}

In the preparatory step, all datafiles are loaded into the R workspace. The SPSS file format is preferred, since it comes with variable and value names that are easy to extract. Next, the so called \emph{technical variables} are identified, corresponding to the survey wave (where applicable), country, year, case (respondent) IDs, and weighting factors. A new variable corresponding to the name of the source table is created, as well as a \texttt{wave} variable in the case of single-wave files.

\hypertarget{step-1-selection-of-source-variables-for-harmonization}{%
\subsection{Step 1: Selection of source variables for harmonization}\label{step-1-selection-of-source-variables-for-harmonization}}

The first step in the harmonization process is the selection of source variables corresponding to the target concepts of interest. The target concepts in this case include trust in different institutions, as many as there are available in the source data.

To map source variables to target variable names, a list of source variables in each dataset is necessary, which I refer to as the \emph{codebook}. This list should include the source variable name, and - since variable names tend to by cryptic - also the source variable label. The frequency distribution of values, as well as value labels, provide additional information in case the variable label is not informative, e.g., to distinguish age in years from age in categories.

The codebook can be generated from labelled datasets, such as those created in SPSS. In R, such a codebook can be created with the package \texttt{labelled} (Larmarange, \protect\hyperlink{ref-labelled}{2019}). The R code for creating the codebook is presented in \protect\hyperlink{appendix1}{Appendix 1}.

The codebook is the exported into a spreadsheet program, where variable labels can be browsed and filtered to select the source variables of interest, and assign appropriate target variable names. The target variable names will be \texttt{t\_trust\_*}, where \texttt{*} is the short version of the institution name, and the prefix \texttt{t\_} identifies the target variable.

\hypertarget{step-2-mapping-source-values-to-target-values}{%
\subsection{Step 2: Mapping source values to target values}\label{step-2-mapping-source-values-to-target-values}}

In the second step, for the variables selected with the codebook, a \emph{crosswalk} template is created, where each value of each variable is in a separate row. Again, the \texttt{labelled} package can be used to obtain the desired format, as shown in \protect\hyperlink{appendix2}{Appendix 2}.

This manual mapping of source values to target values, while mundane, achieves two goals. First, it enables the identification of missing value codes in each variable separately. Missing value codes are not always consistent across variables even in the same dataset, and mistakes in automated recoding of missing values have a potentially large impact on the resulting data. Additionally, researchers might want to treat different missing values differently, i.e., considering the ``don't know'' category as a type of opinion (or rather lack thereof), and the ``refusal'' category as in fact missing.

The second goal is a careful inspection of full labels of the source values, one by one, while consulting the textual survey documentation where necessary. The need to assign each source value a corresponding target value again protests from some mistakes of automation, and is particularly useful in the case of categorical variables with limited standardization across waves and projects, such as education or size of town of residence.

Once the value crosswalk table is ready, it is imported into R, and - for each source variable separately - source and target values are extracted as vectors and used in mapping. The function \texttt{mapvalues} from the \texttt{plyr} package (Wickham, \protect\hyperlink{ref-Wickham2011}{2011}), which maps each value from the source variable vector onto a corresponding value from the target variable vectors, is particularly useful in this regard.

\hypertarget{step-3-recording-characteristics-of-source-items}{%
\subsection{Step 3: Recording characteristics of source items}\label{step-3-recording-characteristics-of-source-items}}

Cross-national surveys differ in almost all impaginable aspects related to the methodology of the survey process, starting from questionnaire development and pretesting, through fieldwork procedures and response or realization rates, the design of individual survey items (Kołczyńska \& Slomczynski, \protect\hyperlink{ref-KolczynskaSlomczynski2018}{2018}), to the quality of documentation (Jabkowski \& Kołczyńska, \protect\hyperlink{ref-Jabkowski2019}{2019}; Marta Kołczyńska \& Schoene, \protect\hyperlink{ref-KolczynskaSchoene2018}{2018}). According to the literature on survey methodology, in particular in the Total Survey Error paradign, the comparability of surveys is reduced whenever the amounts of random and systematic error are different across surveys (Smith, \protect\hyperlink{ref-Smith2018}{2018}). The TSE framework currently counts 35 different sources of error, so the potential for deviations is large.

This is why, while the amount and type (random or systematic) of each type of error in surveys is unknown (and for the most part unknowlable), one approach to dealing with the methodological variation of surveys comprising a single harmonized dataset is to construct additional variables that capture selected properties of the source data or documentation in form of control variables. It is not clear whether including these control variables in substantive analyses improves model estimates or not, and under which cirsumstances, but such methodological controls can at least be used in data exploration.

In the case of questions about trust in institutions, the most easily identifiable differences across items from different projects are in the response scales, in particular in their length and direction (cf.~Kołczyńska \& Slomczynski, \protect\hyperlink{ref-KolczynskaSlomczynski2018}{2018}). All three projects are consistent with their item design, i.e., all trust in institutions items have the same response scales within projects: the European Social Survey has 0-10 ascending scales with labels only for extreme values, the European Quality of Life Survey has 1-10 ascending scales also with only extreme points labelled, and the European Values Study has 4-point descending fully-labelled scales. In this case item designs overlap perfectly with projects, which makes the two effects - the effects of being part of aparticular organizational structure and the effects of the response scale - cannot be analytically distinguished.

Item and other methodological properties can be recorded in a separate table, where the national survey (project*wave*country) is the unit of observation.

\hypertarget{limitations}{%
\section{Limitations}\label{limitations}}

The presented approach naturally has some limitations. Some of them are inherent to the main tool used - the crosswalk. If the source and target schemas do not map into each other very well, the need for the aforementioned ``intelligent flexibility'' (Hillmann \& Westbrooks, \protect\hyperlink{ref-Hillmann2004}{2004}, p. 91) arises. At the same time, crosswalks enable mapping from a single scheme onto another single scheme. Some researchers might want to construct the target variable on the basis of two or more source variables. For example, the target variable ``membership in organizations'' could be constructed on the basis of two target variables corresponding to membership in political organizations and membership in non-political organizations.

\hypertarget{appendices}{%
\section{Appendices}\label{appendices}}

Functions to construct the codebook and values crosswalk, as well as the surveys list, are in Appendix \protect\hyperlink{appendix1}{1}, \protect\hyperlink{appendix2}{2}, and \protect\hyperlink{appendix3}{3}, respectively. The functions rely on the packages \texttt{haven} (Wickham \& Miller, \protect\hyperlink{ref-haven}{2019}) for inporting the data into R, \texttt{labelled} (Larmarange, \protect\hyperlink{ref-labelled}{2019}) for dealing with labels, and the \texttt{tidyverse} (Wickham, \protect\hyperlink{ref-tidyverse}{2017}) suite for data wrangling.

\hypertarget{appendix1}{%
\subsection{Appendix 1: Codebook from labelled data in R}\label{appendix1}}

\scriptsize

\begin{Shaded}
\begin{Highlighting}[]
\KeywordTok{library}\NormalTok{(tidyverse)}
\KeywordTok{library}\NormalTok{(haven)}
\KeywordTok{library}\NormalTok{(labelled)}

\NormalTok{create_codebook <-}\StringTok{ }\ControlFlowTok{function}\NormalTok{(data) \{}
\NormalTok{  var_labels <-}\StringTok{ }\KeywordTok{data.frame}\NormalTok{(}\KeywordTok{cbind}\NormalTok{(}\KeywordTok{names}\NormalTok{(}\KeywordTok{var_label}\NormalTok{(data)), }\KeywordTok{do.call}\NormalTok{(rbind, }\KeywordTok{var_label}\NormalTok{(data)))) }\OperatorTok
\StringTok{    }\KeywordTok{rename}\NormalTok{(}\DataTypeTok{varname =}\NormalTok{ X1, }\DataTypeTok{varlabel =}\NormalTok{ X2)}
  
\NormalTok{  freqs <-}\StringTok{ }\KeywordTok{lapply}\NormalTok{(data, }\ControlFlowTok{function}\NormalTok{(x) \{ }\KeywordTok{return}\NormalTok{(questionr}\OperatorTok{::}\KeywordTok{freq}\NormalTok{(x)) \}) }\OperatorTok
\StringTok{    }\KeywordTok{keep}\NormalTok{(}\ControlFlowTok{function}\NormalTok{(x) }\KeywordTok{nrow}\NormalTok{(x) }\OperatorTok{<}\StringTok{ }\DecValTok{1000}\NormalTok{) }\OperatorTok
\StringTok{    }\KeywordTok{do.call}\NormalTok{(rbind, .) }\OperatorTok\StringTok{ }
\StringTok{    }\NormalTok{tibble}\OperatorTok{::}\KeywordTok{rownames_to_column}\NormalTok{(}\DataTypeTok{var =} \StringTok{"varname_value"}\NormalTok{) }\OperatorTok
\StringTok{    }\KeywordTok{mutate}\NormalTok{(}\DataTypeTok{varname =} \KeywordTok{gsub}\NormalTok{(}\StringTok{"(.+?)(}\CharTok{\textbackslash{}\textbackslash{}}\StringTok{..*)"}\NormalTok{, }\StringTok{"}\CharTok{\textbackslash{}\textbackslash{}}\StringTok{1"}\NormalTok{, varname_value),}
           \DataTypeTok{value =} \KeywordTok{gsub}\NormalTok{(}\StringTok{"^[^.]*."}\NormalTok{,}\StringTok{""}\NormalTok{,varname_value)) }\OperatorTok
\StringTok{    }\KeywordTok{group_by}\NormalTok{(varname) }\OperatorTok
\StringTok{    }\KeywordTok{mutate}\NormalTok{(}\DataTypeTok{npos =} \KeywordTok{row_number}\NormalTok{(),}
           \DataTypeTok{value_n =} \KeywordTok{paste}\NormalTok{(value, n, }\DataTypeTok{sep =} \StringTok{": "}\NormalTok{)) }\OperatorTok
\StringTok{    }\KeywordTok{select}\NormalTok{(varname, value_n, npos) }\OperatorTok
\StringTok{    }\KeywordTok{spread}\NormalTok{(npos, value_n) }\OperatorTok
\StringTok{    }\KeywordTok{mutate_at}\NormalTok{(}\KeywordTok{vars}\NormalTok{(}\OperatorTok{-}\NormalTok{varname), }\KeywordTok{funs}\NormalTok{(}\KeywordTok{ifelse}\NormalTok{(}\KeywordTok{is.na}\NormalTok{(.), }\StringTok{""}\NormalTok{, .))) }\OperatorTok
\StringTok{    }\KeywordTok{unite}\NormalTok{(}\StringTok{"valfreqs"}\NormalTok{, }\KeywordTok{c}\NormalTok{(}\DecValTok{2}\OperatorTok{:}\KeywordTok{ncol}\NormalTok{(.)), }\DataTypeTok{sep =} \StringTok{"}\CharTok{\textbackslash{}n}\StringTok{"}\NormalTok{) }\OperatorTok
\StringTok{    }\KeywordTok{mutate}\NormalTok{(}\DataTypeTok{valfreqs =} \KeywordTok{sub}\NormalTok{(}\StringTok{"}\CharTok{\textbackslash{}\textbackslash{}}\StringTok{s+$"}\NormalTok{, }\StringTok{""}\NormalTok{, valfreqs))}
  
  \KeywordTok{full_join}\NormalTok{(var_labels, freqs, }\DataTypeTok{by =} \StringTok{"varname"}\NormalTok{)}
\NormalTok{\}}
\end{Highlighting}
\end{Shaded}

\normalsize

\hypertarget{appendix2}{%
\subsection{Appendix 2: Values crosswalk}\label{appendix2}}

\scriptsize

\begin{Shaded}
\begin{Highlighting}[]
\NormalTok{create_cwt <-}\StringTok{ }\ControlFlowTok{function}\NormalTok{(data) \{}
\NormalTok{  var_labels <-}\StringTok{ }\KeywordTok{data.frame}\NormalTok{(}\KeywordTok{cbind}\NormalTok{(}\KeywordTok{names}\NormalTok{(}\KeywordTok{var_label}\NormalTok{(data)), }\KeywordTok{do.call}\NormalTok{(rbind, }\KeywordTok{var_label}\NormalTok{(data)))) }\OperatorTok
\StringTok{    }\KeywordTok{rename}\NormalTok{(}\DataTypeTok{varname =}\NormalTok{ X1, }\DataTypeTok{varlabel =}\NormalTok{ X2)}
  
\NormalTok{  freqs_cwt <-}\StringTok{ }\KeywordTok{lapply}\NormalTok{(data, }\ControlFlowTok{function}\NormalTok{(x) \{ }\KeywordTok{return}\NormalTok{(questionr}\OperatorTok{::}\KeywordTok{freq}\NormalTok{(x)) \}) }\OperatorTok
\StringTok{    }\KeywordTok{keep}\NormalTok{(}\ControlFlowTok{function}\NormalTok{(x) }\KeywordTok{nrow}\NormalTok{(x) }\OperatorTok{<}\StringTok{ }\DecValTok{1000}\NormalTok{) }\OperatorTok
\StringTok{    }\KeywordTok{do.call}\NormalTok{(rbind, .) }\OperatorTok\StringTok{ }
\StringTok{    }\NormalTok{tibble}\OperatorTok{::}\KeywordTok{rownames_to_column}\NormalTok{(}\DataTypeTok{var =} \StringTok{"varname_value"}\NormalTok{) }\OperatorTok
\StringTok{    }\KeywordTok{mutate}\NormalTok{(}\DataTypeTok{varname =} \KeywordTok{gsub}\NormalTok{(}\StringTok{"(.+?)(}\CharTok{\textbackslash{}\textbackslash{}}\StringTok{..*)"}\NormalTok{, }\StringTok{"}\CharTok{\textbackslash{}\textbackslash{}}\StringTok{1"}\NormalTok{, varname_value),}
           \DataTypeTok{value =} \KeywordTok{gsub}\NormalTok{(}\StringTok{"^[^.]*."}\NormalTok{,}\StringTok{""}\NormalTok{,varname_value),}
           \DataTypeTok{value_code =} \KeywordTok{sub}\NormalTok{(}\StringTok{".*}\CharTok{\textbackslash{}\textbackslash{}}\StringTok{[(.+)}\CharTok{\textbackslash{}\textbackslash{}}\StringTok{].*"}\NormalTok{, }\StringTok{"}\CharTok{\textbackslash{}\textbackslash{}}\StringTok{1"}\NormalTok{, varname_value, }\DataTypeTok{perl =} \OtherTok{TRUE}\NormalTok{),}
           \DataTypeTok{value_code =} \KeywordTok{ifelse}\NormalTok{(}\KeywordTok{str_sub}\NormalTok{(varname_value, }\DecValTok{-2}\NormalTok{, }\DecValTok{-1}\NormalTok{) }\OperatorTok{==}\StringTok{ "NA"}\NormalTok{, }\StringTok{"NA"}\NormalTok{, value_code),}
           \DataTypeTok{value_code =} \KeywordTok{ifelse}\NormalTok{(}\KeywordTok{gsub}\NormalTok{(}\StringTok{" "}\NormalTok{, }\StringTok{""}\NormalTok{, }\DataTypeTok{fixed =} \OtherTok{TRUE}\NormalTok{, varname_value) }\OperatorTok{==}\StringTok{ }\NormalTok{varname_value, }
                               \KeywordTok{gsub}\NormalTok{(}\StringTok{"^[^.]*."}\NormalTok{,}\StringTok{""}\NormalTok{,varname_value), value_code)) }\OperatorTok
\StringTok{    }\KeywordTok{group_by}\NormalTok{(varname) }\OperatorTok
\StringTok{    }\KeywordTok{mutate}\NormalTok{(}\DataTypeTok{npos =} \KeywordTok{row_number}\NormalTok{(),}
           \DataTypeTok{value_n =} \KeywordTok{paste}\NormalTok{(value, n, }\DataTypeTok{sep =} \StringTok{": "}\NormalTok{)) }\OperatorTok
\StringTok{    }\KeywordTok{select}\NormalTok{(npos, varname, value_n, value, value_code)}
  
  \KeywordTok{full_join}\NormalTok{(var_labels, freqs_cwt, }\DataTypeTok{by =} \StringTok{"varname"}\NormalTok{)}
\NormalTok{\}}
\end{Highlighting}
\end{Shaded}

\normalsize

\hypertarget{appendix3}{%
\subsection{Appendix 3: Metadata table}\label{appendix3}}

\scriptsize

\begin{Shaded}
\begin{Highlighting}[]
\CommentTok{# This function can be run after the following variables in the source data have been constructed:}
\CommentTok{# t_project for the project abbreviation;}
\CommentTok{# t_round for the project round;}
\CommentTok{# t_country for the country code;}
\CommentTok{# t_year for the survey year.}

\NormalTok{create_surveys_list <-}\StringTok{ }\ControlFlowTok{function}\NormalTok{(data_frame_vector) \{}
  
\NormalTok{  surveys_list <-}\StringTok{ }\KeywordTok{list}\NormalTok{()}
  
  \ControlFlowTok{for}\NormalTok{ (i }\ControlFlowTok{in} \DecValTok{1}\OperatorTok{:}\KeywordTok{length}\NormalTok{(data_frame_vector)) \{}
    
\NormalTok{    surveys_list[[i]] <-}\StringTok{ }\KeywordTok{eval}\NormalTok{(}\KeywordTok{parse}\NormalTok{(}\DataTypeTok{text =}\NormalTok{ data_frame_vector[i])) }\OperatorTok
\StringTok{      }\KeywordTok{mutate}\NormalTok{(}\DataTypeTok{t_survey =} \KeywordTok{paste}\NormalTok{(t_project, t_round, t_country, }\DataTypeTok{sep =} \StringTok{""}\NormalTok{)) }\OperatorTok
\StringTok{      }\KeywordTok{count}\NormalTok{(t_survey, t_year)}
\NormalTok{  \}}
  
  \KeywordTok{do.call}\NormalTok{(rbind, surveys_list)}
\NormalTok{\}}
\end{Highlighting}
\end{Shaded}

\normalsize

\hypertarget{references}{%
\section{References}\label{references}}

\renewcommand{\baselinestretch}{1}\normalsize

\parindent -0in
\leftskip 0.2in
\parindent -0.2in

\hypertarget{refs}{}
\leavevmode\hypertarget{ref-Bekkers2015}{}%
Bekkers, R., Meer, T. van der, Uslaner, E., Wu, Z., Wit, A. de, \& Blok, L. de. (2015). Harmonized Trust Database. \url{https://doi.org/10.17605/OSF.IO/QFV76}

\leavevmode\hypertarget{ref-Buckheit1995}{}%
Buckheit, J. B., \& Donoho, D. L. (1995). WaveLab and Reproducible Research. In A. Antoniadis \& G. Oppenheim (Eds.), \emph{Wavelets and statistics} (pp. 55--81). New York, NY: Springer New York. \url{https://doi.org/10.1007/978-1-4612-2544-7_5}

\leavevmode\hypertarget{ref-Burkhauser2005}{}%
Burkhauser, R. V., \& Lillard, D. R. (2005). The contribution and potential of data harmonization for cross-national comparative research. \emph{Journal of Comparative Policy Analysis: Research and Practice}, \emph{7}(4), 313--330. \url{https://doi.org/10.1080/13876980500319436}

\leavevmode\hypertarget{ref-Christmann2018}{}%
Christmann, P. (2018). Economic performance, quality of democracy and satisfaction with democracy. \emph{Electoral Studies}, \emph{53}(April 2017), 79--89. \url{https://doi.org/10.1016/j.electstud.2018.04.004}

\leavevmode\hypertarget{ref-Christmann2018a}{}%
Christmann, P., \& Torcal, M. (2018). The Effects of Government System Fractionalization on Satisfaction with Democracy. \emph{Political Science Research and Methods}, \emph{6}(3), 593--611. \url{https://doi.org/10.1017/psrm.2017.23}

\leavevmode\hypertarget{ref-Curran2009}{}%
Curran, P. J., \& Hussong, A. M. (2009). Integrative Data Analysis: The Simultaneous Analysis of Multiple Data Sets. \emph{Psychological Methods}, \emph{14}(2), 81--100. \url{https://doi.org/10.1037/a0015914}

\leavevmode\hypertarget{ref-ESSEuropeanSocialSurvey2016b}{}%
ESS (European Social Survey). (2016). ESS Round 8 Source Questionnaire. London: ESS ERIC Headquarters c/o City University London.

\leavevmode\hypertarget{ref-Fortier2017a}{}%
Fortier, I., Raina, P., Van den Heuvel, E. R., Griffith, L. E., Craig, C., Saliba, M., \ldots{} Burton, P. (2017). Maelstrom Research guidelines for rigorous retrospective data harmonization. \emph{International Journal of Epidemiology}, \emph{46}(1), 103--115. \url{https://doi.org/10.1093/ije/dyw075}

\leavevmode\hypertarget{ref-Granda2016}{}%
Granda, P., \& Blasczyk, E. (2016). Data Harmonization. In \emph{Guidelines for best practice in cross-cultural surveys} (pp. 617--635). Ann Arbor, MI: Survey Research Center, Institute for Social Research, University of Michigan. Retrieved from \url{http://www.ccsg.isr.umich.edu/}

\leavevmode\hypertarget{ref-Hillmann2004}{}%
Hillmann, D. I., \& Westbrooks, E. L. (2004). \emph{Metadata in Practice} (p. 285). American Library Association.

\leavevmode\hypertarget{ref-Jabkowski2019}{}%
Jabkowski, P., \& Kołczyńska, M. (2019). Survey practices in Europe: Analysis of methodological documentation from 1537 surveys in five cross-national projects, 1981-2017. Unpublished draft.

\leavevmode\hypertarget{ref-Klassen2018}{}%
Klassen, A. J. (2018). \emph{Human Understanding Measured Across National ( HUMAN) Surveys. Codebook for Respondent Data} (No. 15 February). Harvard Dataverse, V1. \url{https://doi.org/10.7910/DVN/QLKR85}

\leavevmode\hypertarget{ref-Kolczynska2019}{}%
Kołczyńska, M. (2019). From Cross-national ex post Survey Harmonization to Substantive Analyses: A Roadmap and Empirical Illustration of Micro- and Macro-level Determinants of Protest Participation. Unpublished draft.

\leavevmode\hypertarget{ref-KolczynskaSlomczynski2018}{}%
Kołczyńska, M., \& Slomczynski, K. M. (2018). Item Metadata as Controls for Ex Post Harmonization of International Survey Projects. In T. P. Johnson, B.-E. Pennell, I. A. L. Stoop, \& B. Dorer (Eds.), \emph{Advances in comparative survey methods: Multinational, multiregional, and multicultural contexts (3MC)} (pp. 1011--1034). Wiley.

\leavevmode\hypertarget{ref-labelled}{}%
Larmarange, J. (2019). \emph{labelled: Manipulating Labelled Data}. Retrieved from \url{https://cran.r-project.org/package=labelled}

\leavevmode\hypertarget{ref-Liu2019}{}%
Liu, D. M., \& Salganik, M. J. (2019). \emph{Successes and struggles with computational reproducibility: Lessons from the Fragile Families Challenge} (pp. 1--56). \url{https://doi.org/10.31235/osf.io/g3pdb}

\leavevmode\hypertarget{ref-KolczynskaSchoene2018}{}%
Marta Kołczyńska, \& Schoene, M. (2018). Survey Data Harmonization and the Quality of Data Documentation in Cross‐national Surveys. In T. P. Johnson, B.-E. Pennell, I. A. L. Stoop, \& B. Dorer (Eds.), \emph{Advances in comparative survey methods: Multinational, multiregional, and multicultural contexts (3MC)} (pp. 963--984). Wiley.

\leavevmode\hypertarget{ref-Mauk2019}{}%
Mauk, M. (2019). Disentangling an Elusive Relationship: How Democratic Value Orientations Affect Political Trust in Different Regimes. \emph{Political Research Quarterly}. \url{https://doi.org/10.1177/1065912919829832}

\leavevmode\hypertarget{ref-Powako2016}{}%
Powałko, P., \& Kołczyńska, M. (2016). Working with Data in the Cross-National Survey Harmonization Project: Outline of Programming Decisions. \emph{International Journal of Sociology}, \emph{46}(1), 73--80. \url{https://doi.org/10.1080/00207659.2016.1130433}

\leavevmode\hypertarget{ref-Sandberg2018}{}%
Sandberg, B., \& Bekkers, R. (2018). \emph{Harmonized Trust Database Codebook, Version 1.3}. Global Trust Research Consortium. Retrieved from \url{https://osf.io/92r5z/}

\leavevmode\hypertarget{ref-Slomczynski2018}{}%
Slomczynski, K. M., \& Tomescu-Dubrow, I. (2018). Basic Principles of Survey Data Recycling. In T. P. Johnson, B.-E. Pennell, I. A. L. Stoop, \& B. Dorer (Eds.), \emph{Advances in comparative survey methods: Multinational, multiregional, and multicultural contexts (3MC)} (pp. 937--962). Wiley. \url{https://doi.org/10.1002/9781118884997.ch43}

\leavevmode\hypertarget{ref-Smith2018}{}%
Smith, T. W. (2018). Improving Multinational, Multiregional, and Multicultural (3MC) Comparability Using the Total Survey Error (TSE) Paradigm. In \emph{Advances in comparative survey methods} (pp. 13--43). Hoboken, NJ, USA: John Wiley \& Sons, Inc. \url{https://doi.org/10.1002/9781118884997.ch2}

\leavevmode\hypertarget{ref-Slomczynskietal2016}{}%
Słomczyński, K. M., Tomescu-Dubrow, I., \& Jenkins, J. C. (2016). \emph{Democratic Values and Protest Behavior. Harmonization of Data from International Survey Projects.} Warsaw: IFiS Publishers.

\leavevmode\hypertarget{ref-Wickham2011}{}%
Wickham, H. (2011). The Split-Apply-Combine Strategy for Data Analysis. \emph{Journal of Statistical Software}, \emph{40}(1), 1--29. Retrieved from \url{http://www.jstatsoft.org/v40/i01/}

\leavevmode\hypertarget{ref-tidyverse}{}%
Wickham, H. (2017). \emph{tidyverse: Easily Install and Load the 'Tidyverse'}. Retrieved from \url{https://cran.r-project.org/package=tidyverse}

\leavevmode\hypertarget{ref-haven}{}%
Wickham, H., \& Miller, E. (2019). \emph{haven: Import and Export 'SPSS', 'Stata' and 'SAS' Files}. Retrieved from \url{https://cran.r-project.org/package=haven}

\leavevmode\hypertarget{ref-Wolf2016}{}%
Wolf, C., Schneider, S. L., Behr, D., \& Joye, D. (2016). Harmonizing Survey Questions Between Cultures and Over Time. In \emph{The sage handbook of survey methodology} (pp. 502--524). SAGE. \url{https://doi.org/10.4135/9781473957893}


\end{document}
